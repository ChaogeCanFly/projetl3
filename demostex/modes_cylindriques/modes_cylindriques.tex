\documentclass[11pt]{article}
% \usepackage[french]{babel}
\usepackage[utf8]{inputenc}
\usepackage{tikz}
\usepackage{amsmath}
\usepackage{amsfonts}
\usepackage[top=3cm,right=2cm,bottom=2cm,left=2cm]{geometry}

\renewcommand{\phi}\varphi
\newcommand{\J}{\mathrm{J}}

\title{Modes dans un guide cyclindrique}
\author{C. Laurent \& M. Gaborit}
\date{Projet L3}

\begin{document}
\maketitle

On cherche à définir les modes dans un guide d'onde cylindrique.

\begin{figure}
    \centering
\begin{tikzpicture}[>=latex,scale=0.6]
    % cercle
    \draw (0,0) circle (5);

    % axes
    \draw (-6,0) -- (6,0);
    \draw (0,-6) -- (0,6);
    \draw (0,0) circle (0.4);
    \draw[fill=black] (0,0) circle (0.1);


    % vecteur
    \draw[->] (0,0) -- (30:4) node[midway,left,above] {$r$};
    \draw (2,0) arc (0:30:2);
    \draw (15:2) node[right] {$\theta$};
    \draw[fill=black](30:4) circle (.1) node[right] {$M$};

    % Rayon
    \draw (0,0) -- (40:-6) node[midway,below,right] {$R$};
\end{tikzpicture}
\caption{\label{cyl1:fig} Coupe du guide}
\end{figure}

\subsection*{Position du problème}

On se place en coordonnées cylindriques (voir figure~\ref{cyl1:fig}, donc sur un domaine $D =
\{\forall(r,\theta)\in[0,1]\times[-\pi;\pi]\}$), en régime harmonique.

\paragraph{Equation de propagation} Ainsi, on cherchera à résoudre l'équation dite d'Helmholtz~\eqref{cyl1:helm} :

\begin{equation}
\Delta p  + k^2p = 0 \label{cyl1:helm}
\end{equation}

Avec la condition limite suivante :

\begin{equation}
    \left.\frac{\partial p}{\partial n}\right|_{r=R} = 0\label{cyl1:CL1}
\end{equation}

\paragraph{Laplacien} En coordonnées cyclindriques :

$$\Delta \bullet = \frac{1}{r}\frac{\partial}{\partial r}\left[r\frac{\partial\bullet}{\partial r}\right] +
\frac{1}{r^2}\frac{\partial^2\bullet}{\partial\theta^2}+\frac{\partial^2\bullet}{\partial z^2}$$

En coordonnées cylindriques, on a $p(t,r,\theta,z)$. On choisira de séparer les variables :

\begin{equation}
    p(t,r,\theta,z) = \psi(r)\phi(\theta)e^{j(\omega t+k^{(z)}z)} \label{cyl1:eq_p}
\end{equation}

On réécrit donc~\eqref{cyl1:helm} :

\begin{equation*}
\begin{array}{c}
\frac{1}{r}\frac{\partial}{\partial r}\left[r\frac{\partial\psi}{\partial r}\right]\phi +
\frac{1}{r^2}\frac{\partial^2\phi}{\partial\theta^2}\psi+k^2\psi\phi = 0\\
%
\frac{1}{r\psi}\frac{\partial}{\partial r}\left[r\frac{\partial\psi}{\partial r}\right] +
\frac{1}{\phi r^2}\frac{\partial^2\phi}{\partial\theta^2}+k^2 = 0\\
%
\frac{1}{r\psi}\left[r\frac{\partial^2\psi}{\partial r^2}+\frac{\partial\psi}{\partial r}\right] +
\frac{1}{\phi r^2}\frac{\partial^2\phi}{\partial\theta^2}+k^2 = 0\\
%
\frac{r}{\psi}\left[r\frac{\partial^2\psi}{\partial r^2}+\frac{\partial\psi}{\partial r}\right] + k^2r^2 +
\frac{1}{\phi}\frac{\partial^2\phi}{\partial\theta^2} = 0\\
\end{array}
\end{equation*}

On a alors :
\begin{equation}
-\left\{\frac{r}{\psi}\left[r\frac{\partial^2\psi}{\partial r^2}+\frac{\partial\psi}{\partial r}\right] + k^2r^2\right\} =
\frac{1}{\phi}\frac{\partial^2\phi}{\partial\theta^2} = -\mu ~~;~~ \mu < 0\label{cyl1:sepvar}\\
\end{equation}

\paragraph{Condition de recollement} En $\theta=\pm\pi$ (aux limites du domaine d'étude) la fonction $\phi(\theta)$ doit
avoir une même valeur ainsi, nous avons la condition de recollement suivante :

\begin{equation}
\begin{cases}
    \phi(-\pi) = \phi(\pi)&\\
    \frac{\partial\phi(-\pi)}{\partial \theta} = \frac{\partial\phi(\pi)}{\partial \theta}&
\end{cases}\label{cyl1:CL2}
\end{equation}

Ainsi, en reprenant le terme de droite de~\eqref{cyl1:sepvar}, on retrouve :

\begin{equation}
    \frac{\partial^2\phi}{\partial\theta^2} = - \mu\phi \Rightarrow \phi(\theta) = A\cos(\theta\sqrt{\mu}) +
    B\sin[\theta\sqrt{\mu}] \label{cyl1:eq_phi}
\end{equation}

En appliquant la condition limite~\eqref{cyl1:CL2}, on obtient :

\begin{equation}
    \begin{array}{rrcl}
        &\phi(-\pi) & = & \phi(\pi)\\
        \Leftrightarrow & A\cos(-\pi\sqrt{\mu}) + B\sin(-\pi\sqrt{\mu}) & = & A\cos(\pi\sqrt{\mu}) + B\sin(\pi\sqrt{\mu}) \\
        \Leftrightarrow & 2B\sin(\pi\sqrt{\mu}) & = & 0\\
        \Leftrightarrow & \sin(\pi\sqrt{\mu}) & = & 0\\
        \Leftrightarrow & \pi\sqrt{\mu} & = & n\pi ~~;~~ n\in\mathbb{N}\\
        \Leftrightarrow & \sqrt{\mu} & \equiv & n
    \end{array}        
\end{equation}

On s'occupe maintenant de la deuxième partie de l'équation~\eqref{cyl1:sepvar}.

\begin{equation}
    \frac{r}{\psi}\left[r\frac{\partial^2\psi}{\partial r^2}+\frac{\partial\psi}{\partial r}\right] + k^2r^2 =
    \sqrt{\mu} = n^2 \Leftrightarrow r^2\frac{\partial^2\psi}{\partial r^2} + r\frac{\partial\psi}{\partial r} +
    (r^2k^2-n^2)\psi = 0 \label{cyl1:eq_psi}
\end{equation}

\paragraph{Condition de non-divergence} On admet que :

\begin{equation}
    \lim_{r\to 0} \psi_n(r) < \infty \label{cyl1:CL3}
\end{equation}

Les solutions de l'équation ~\eqref{cyl1:eq_psi} sont donc de la forme :

\begin{equation}
    \psi_{n,m}(r) = A_{n,m}\J_n(\alpha_{n,m}r) + \underbrace{B_{n,m}\mathrm{K}_n(\alpha{n,m}r)}_{\eqref{cyl1:CL3}
        \Rightarrow B_{n,m}
    = 0} \label{cyl1:eq_psi2}
\end{equation}

On pose par ailleurs : $$k_{n,m}^{(r)} = (\alpha_{n,m})^2$$ ; les $\alpha_{n,m}$ correspondent aux zéros des fonctions
de bessel $\J_n$ de première espèce et d'ordre $n$.

\paragraph{Solution particulière}

On recombine les équations~\eqref{cyl1:eq_p},~\eqref{cyl1:eq_phi} et ~\eqref{cyl1:eq_psi2} :

\begin{equation}
    p_{n,m}(r,\theta,t) = \J_n(\alpha_{n,m})\left[a_{n,m}\cos(n\theta)+b_{n,m}\sin(n\theta)\right]e^{j(\omega t - k_{n,m}^{(z)}z)} \label{cyl1:pnm}
\end{equation}

On notera d'ailleurs que $$k_{n,m}^{(z)}\sqrt{k^2-(\alpha_{n,m})^2}$$

\paragraph{Solution Générale} En sommant ~\eqref{cyl1:pnm} sur $n$ et $m$, on obtient la solution générale suivante.

\begin{equation}
    p(r,\theta,t) = \sum_{n=0}^\infty\sum_{m=0}^\infty
    \J_n(\alpha_{n,m})\left[a_{n,m}\cos(n\theta)+b_{n,m}\sin(n\theta)\right]e^{j(\omega t - k_{n,m}^{(z)}z)}
    \label{cyl1:ptot}
\end{equation}
\end{document}

